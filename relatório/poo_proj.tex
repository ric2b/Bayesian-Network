\documentclass[10pt]{article}

\usepackage[portuguese]{babel}
\usepackage[utf8]{inputenc}
\usepackage{amsmath}
\usepackage{graphicx}
\usepackage{float}
\usepackage{subfig}
\usepackage{fixltx2e}
\usepackage[bottom]{footmisc}
\usepackage{listings}
\usepackage{color} 
\usepackage[usenames,dvipsnames]{xcolor}
\usepackage[colorinlistoftodos]{todonotes}
\usepackage[font=footnotesize]{caption}

\definecolor{keywordcolor}{rgb}{0,0.4,0.7}
\definecolor{commentcolor}{rgb}{0.4,0.4,0.4} 	
\definecolor{mygray}{rgb}{0.5,0.5,0.5} 	% line counter color
\definecolor{mymauve}{rgb}{0.90,0.25,0.47}	% string color
\definecolor{codebackground}{rgb}{0.95,0.95,0.95} 

\lstset{ %
	backgroundcolor=\color{codebackground},   % choose the background color; you must add \usepackage{color} or \usepackage{xcolor}
	basicstyle=\ttfamily \tiny,        % the size of the fonts that are used for the code
	breakatwhitespace=false,         % sets if automatic breaks should only happen at whitespace
	breaklines=true,                 % sets automatic line breaking
	captionpos=b,                    % sets the caption-position to bottom
	commentstyle=\color{commentcolor},    % comment style
	deletekeywords={...},            % if you want to delete keywords from the given language
	escapeinside={\%*}{*)},          % if you want to add LaTeX within your code
	extendedchars=true,              % lets you use non-ASCII characters; for 8-bits encodings only, does not work with UTF-8
	keepspaces=true,                 % keeps spaces in text, useful for keeping indentation of code (possibly needs columns=flexible)
	keywordstyle=\color{keywordcolor},       % keyword style
	numbers=left,                    % where to put the line-numbers; possible values are (none, left, right)
	numbersep=5pt,                   % how far the line-numbers are from the code
	numberstyle=\tiny\color{mygray}, % the style that is used for the line-numbers
	rulecolor=\color{black},         % if not set, the frame-color may be changed on line-breaks within not-black text (e.g. comments (green here))
	showspaces=false,                % show spaces everywhere adding particular underscores; it overrides 'showstringspaces'
	showstringspaces=false,          % underline spaces within strings only
	showtabs=false,                  % show tabs within strings adding particular underscores
	stepnumber=1,                    % the step between two line-numbers. If it's 1, each line will be numbered
	%stringstyle=\color{mymauve},     % string literal style
	%identifierstyle=\color{mymauve},
	tabsize=2                       % sets default tabsize to 2 spaces
}

\setcounter{tocdepth}{1}

\numberwithin{equation}{section}

\linespread{1.3}
\usepackage{indentfirst}
\usepackage[top=2cm, bottom=2cm, right=2.25cm, left=2.25cm]{geometry}
\addto\captionsportuguese{\renewcommand{\contentsname}{Índice}}

\begin{document}

\begin{titlepage}
\begin{center}

\hfill \break
\hfill \break

\includegraphics[width=0.3\textwidth]{./logo}~\\[1cm]

\textsc{\LARGE Instituto Superior Técnico}\\[0.25cm]
\textsc{\Large Mestrado Integrado em Engenharia Electrotécnica e de Computadores}\\[1.8cm]
\textsc{\huge Programação Orientada a Objectos}\\[0.25cm]

{\huge \bfseries Aprendizagem de redes dinâmicas de Bayes \\[1cm]}

\begin{tabular}{ l l }
Maria Margarida Dias dos Reis & \hspace{2mm} n.º 73099 \\
Ricardo Filipe Amendoeira & \hspace{2mm} n.º 73373 \\
David Romão Fialho & \hspace{2mm} n.º 73530
\end{tabular}

\vfill

{\large Lisboa, 18 de Maio de 2015} 

\end{center}
\end{titlepage}

\pagenumbering{gobble}
\clearpage

\tableofcontents
\pagebreak

\clearpage
\pagenumbering{arabic}

\section{Introdução}

Com este projecto pretende-se utilizar redes dinâmicas de Bayes (DBN) para modelar uma série multivariante no tempo. \todo{continuar isto}

\section{Decisões de Projecto}

Para se definir a estrutura do projecto optou-se, à partida, por tentar implementar uma estrutura modular e reutilizável, ou seja, algo que funcione não apenas com o que se pretende elaborar e com os requisitos a cumprir mas sim para casos genéricos. Assim, as \textit{features} que foram projectadas tendo como base uma \textit{framework} reutilizável são:

\begin{itemize}
	\item \textit{Bayesian Network} (BN) e \textit{Dinamic Bayesian Network} (DBN);
	\vspace{-2.5mm}
	\item grafo;
	\vspace{-2.5mm}
	\item operações; 
	\vspace{-2.5mm}
	\item \textit{score};
	\vspace{-2.5mm}
	\item critério de paragem do algoritmo GHC;
	\vspace{-2.5mm}
	\item número de pais de um dado nó.
\end{itemize}

É de referir que, apesar da rapidez de computação ser um critério relativamente importante para um programa deste género, decidiu-se que ter uma solução que providencia uma \textit{framework} extensível e reutilizável para a aprendizagem de DBNs é ainda mais importante, quando se considera o âmbito da cadeira na qual o projecto está inserido.

\section{Testes Efectuados}

\subsection{Inferências}

Inicialmente, para se verificar o funcionamento das inferências optou-se por verificar que a soma das probabilidades obtidas para os valores futuros possíveis da variável aleatória que se está a inferir dá próximo de 1, isto porque, obrigatoriamente, a variável aleatória toma no futuro um valor dos possíveis do seu \textit{range}. Este teste foi feito com recurso ao grafo apresentado de seguida, que se forçou na execução do programa.

\begin{figure}[H]
	\centering
	\includegraphics[keepaspectratio=true, scale=0.17]{teoricas/grafoforcado}
	\caption{Grafo da rede de transição utilizada para testar as inferências.}
	\vspace{-0.8em}
\end{figure}

Assumindo todas as variáveis aleatórias como binárias, os valores obtidos para as probabilidades das três variáveis do futuro para os seus dois valores possíveis apresentam-se na seguinte tabela.

\begin{table}[H]
	\centering
	\caption{Probabilidades obtidas para os valores das variáveis aleatórias no futuro.}
	\vspace{-1.5mm}
	\includegraphics[keepaspectratio=true, scale=0.30]{tabelas/probabilidades}
\end{table}

\vspace{-1.5mm}
Como se pode ver, todas as probabilidades têm um valor muito próximo de 1, sendo normal que algumas sejam de facto maiores, uma vez que são calculadas com recurso a estimativas.

Para verificar as inferências que o programa obtém optou-se por usar o teste \#2 fornecido na página da cadeira como ficheiro de treino para se efectuar a aprendizagem da rede de Bayes. O ficheiro de teste construído corresponde aos valores do instante de tempo $t = 0$ do ficheiro de treino, sendo que, algumas das inferências, devem tomar valores próximos dos do instante de tempo $t = 1$ do ficheiro de treino:

\begin{lstlisting}
A, B, C, D, E, F, G
3, 3, 1, 0, 1, 1, 2
2, 3, 0, 3, 3, 1, 0
3, 1, 0, 1, 1, 2, 3
\end{lstlisting}

Considerando a construção da rede de Bayes sem \textit{random restarts} aquando da aplicação do algoritmo GHC tem-se a seguinte execução de programa para a inferência de todas as variáveis aleatórias. De notar que o teste apresentado de seguida foi executado nos computadores do laboratório.

\begin{lstlisting}
Parameters: train-data-2.csv test-data-2-TRAIN.csv LL 0
Building DBN: 28.934149246 seconds
Initial network:
=== Structure connectivity
A : C
B :
C : D
D : G
E : A
F : A
G :
=== Scores
LL Score: -4.754887502163469
MDL Score: -131.55188755985597
Transition network:
=== Inter-slice connectivity
A : F G
B : F G
C : F
D : C F G
E : F
F :
G :
=== Intra-slice connectivity
A : C
B : D
C : E D
D :
E : B D
F : E D G
G : A B E
=== Scores
LL Score: -4233.21662434474
MDL Score: -33449.233345448425
Performing inference:
-> instance 1: 3, 3, 1, 0, 1, 1, 0
-> instance 2: 2, 4, 0, 3, 1, 1, 0
-> instance 3: 3, 2, 0, 2, 0, 1, 0
Infering with DBN: 64.80672281400001 seconds
\end{lstlisting}

Analisando os valores pretendidos e aqueles que foram de facto obtidos verifica-se uma eficácia de 61.91\%, um valor que se considera aceitável e, tendo em conta, que as somas das probabilidades já foram verificadas como estando próximas de 1 assume-se que as inferências estão correctas.

\section{Conclusão}

\end{document}